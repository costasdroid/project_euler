\documentclass[12pt]{scrartcl}
\usepackage[utf8]{inputenc}
\usepackage{amsmath}

\usepackage{listings,xcolor}

\lstset{language=Mathematica}

\begin{document}

According to Heron the area of a triangle with sides $a$,$b$,$c$ is

\begin{align*}
    E=\sqrt{s(s-a)(s-b)(s-c)}
\end{align*}

For circles $C_0-C_1-C_8$ the area of the triangle $K_0K_1K_8$ is according to Heron

\begin{align*}
    s & =\frac{1}{2} \left( r_0+r_1+r_1+r_8+r_8+r_0 \right) \\
    s & =r_0+r_1+r_8                                        \\
    E & =\sqrt{r_0r_1r_8(r_0+r_1+r_8)}
\end{align*}

with $r_k=d^k\cdot r_0$ and $r_0$=1

\begin{align*}
    E & =\sqrt{d^9(1+d+d^8)} \\
    E & =d^4\sqrt{d+d^2+d^9}
\end{align*}

The sector of $C_0$ has area

\begin{align*}
    A & =\frac{1}{2}r_0^2\theta_0 \\
    A & =\frac{1}{2}\theta_0
\end{align*}

From the cosine rule we have

\begin{align*}
    \cos\theta_0 & =\frac{(r_0+r_1)^2+(r_0+r_8)^2-(r_1+r_8)^2}{2(r_0+r_1)(r_0+r_8)} \\
    \cos\theta_0 & =\frac{(1+d)^2+(1+d^8)^2-(d+d^8)^2}{2(1+d)(1+d^8)}               \\
    \cos\theta_0 & =\frac{1+2d+d^2+1+2d^8+d^{16}-d^2-2d^9-d^{16}}{2(1+d)(1+d^8)}    \\
    \cos\theta_0 & =\frac{2+2d+2d^8-2d^9}{2(1+d)(1+d^8)}                            \\
    \cos\theta_0 & =\frac{1+d+d^8-d^9}{(1+d)(1+d^8)}                                \\
\end{align*}

% Again for the circles $C_0-C_7-C_8$ the area of the triangle $K_0K_7K_8$ is

% \begin{align*}
%     E & =d^7\sqrt{d+d^8+d^9}
% \end{align*}

\end{document}